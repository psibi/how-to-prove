%%%%%%%%%%%%%%%%%%%%%%%%%%%%%%%%%%%%%%%%%
% Author: Sibi <sibi@psibi.in>
%%%%%%%%%%%%%%%%%%%%%%%%%%%%%%%%%%%%%%%%%
\documentclass{article}
\usepackage{graphicx}
\usepackage{verbatim}
\usepackage{amsmath}
\usepackage{amsfonts}
\usepackage{amssymb}
\usepackage{tabularx}
\setlength\parskip{\baselineskip}
\begin{document}
\title{Chapter 4 (Section 4.3)}
\author{Sibi}
\date{\today}
\maketitle
\newpage

\section{Problem 1}
\subsection{Solution (a)}

\begin{itemize}
\item Reflexive
\item Transitive
\item Anti-symmetric (And hence partial order)
\item Not total order because $(a,c) \lor (c,a) \notin R$
\end{itemize}

\subsection{Solution (b)}
\begin{itemize}
\item Reflexive 
\item Transitive
\item Not Anti-symmetric. (Example: $(2,-2)$)
\end{itemize}

\subsection{Solution (c)}
\begin{itemize}
\item Reflexive
\item Transitive
\item Anti symmetric
\item Not total order. (Example: $(-2,2)$)
\end{itemize}

\section{Problem 2}
\subsection{Solution (a)}
\begin{itemize}
\item Reflexive
\item Transitive
\item Anti-symmetric
\item Not total order. Example: $(go, haskell)$
\end{itemize}

\subsection{Solution (b)}
\begin{itemize}
\item Reflexive
\item Transitive
\item Not anti-symmetric (Example: $(aba, aa)$)
\end{itemize}

\subsection{Solution (c)}
\begin{itemize}
\item Reflexive
\item Transitive
\item If there is two country with same population, then it is not anti-symmetric.
\end{itemize}

\end{document}
