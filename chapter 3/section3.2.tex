%%%%%%%%%%%%%%%%%%%%%%%%%%%%%%%%%%%%%%%%%
% Author: Sibi <sibi@psibi.in>
%%%%%%%%%%%%%%%%%%%%%%%%%%%%%%%%%%%%%%%%%
\documentclass{article}
\usepackage{graphicx}
\usepackage{verbatim}
\usepackage{amsmath}
\usepackage{amsfonts}
\usepackage{amssymb}
\usepackage{tabularx}
\setlength\parskip{\baselineskip}
\begin{document}
\title{Chapter 3 (Section 3.1)}
\author{Sibi}
\date{\today}
\maketitle
\newpage

\section{Problem 1}

Solution (a)

%% +--------------+--------------+
%% |Givens        |Goals         |
%% +--------------+--------------+
%% |P \implies Q  |P \implies R  |
%% |              |              |
%% |Q \implies R  |              |
%% +--------------+--------------+
% This LaTeX table template is generated by emacs 24.3.1
\begin{tabular}{| >{$}l<{$} | >{$}l<{$} |}
\hline
Givens & Goals \\
\hline
P \implies Q & P \implies R \\
 & \\
Q \implies R & \\
\hline
\end{tabular}


%% +--------------+--------------+
%% |Givens        |Goals         |
%% +--------------+--------------+
%% |P \implies Q  |R             |
%% |              |              |
%% |Q \implies R  |              |
%% |              |              |
%% |P             |              |
%% +--------------+--------------+
% This LaTeX table template is generated by emacs 24.3.1
\begin{tabular}{| >{$}l<{$} | >{$}l<{$} |}
\hline
Givens & Goals \\
\hline
P \implies Q & R \\
 & \\
Q \implies R & \\
 & \\
P & \\
\hline
\end{tabular}

Proof Structure

Suppose $P$.

   [Proof of R goes here]
   
Therefore $P \implies R$.

Proof. Suppose $P$. Since $P \implies Q$, using modus ponens we can
conclude $Q$. Since $Q \implies R$, again from modus ponens we can
conclude R. Therefore, $P \implies R$.

Solution (b)

%% +--------------------+--------------+
%% |Givens              |Goals         |
%% +--------------------+--------------+
%% |\neg R \implies (P  |P \implies (Q |
%% |\implies \neg Q)    |\implies R)   |
%% +--------------------+--------------+
% This LaTeX table template is generated by emacs 24.3.1
\begin{tabular}{| >{$}l<{$} | >{$}l<{$} |}
\hline
Givens & Goals \\
\hline
\neg R \implies (P & P \implies (Q \\
\implies \neg Q) & \implies R) \\
\hline
\end{tabular}

$ \neg R \implies (P \implies \neg Q)$ is equivalent to $(P \land Q)
\implies R$

Proof. Suppose P. Suppose Q. Since $ \neg R \implies (P \implies \neg
Q) $ is equivalent to $(P \land Q) \implies R$, we can conclude R.
Therefore, $P \implies (Q \implies R)$.

\section{Problem 2}

Solution(a)

Proof. Suppose P. Since $P \implies Q$, we can conclude Q. Since $R
\implies \neg Q$ is equivalent to $Q \implies \neg R$, we can conclude
$\neg R$ as Q is true. Therefore, $P \implies \neg R$ is true.

Solution (b)

Proof. Suppose Q. We will prove by contradiction. Let $Q \implies \neg
P$ be true. But $\neg P$ is true but our given $P$ is also true which
leads us to a contradiction. Therefore, $\neg(Q \implies \neg P)$ is
true.

\section{Problem 3}

Proof. Suppse $x \in A$. Since $A \subseteq C$, it follows $x \in C$.
But B and C are disjoint, so $x \notin B$.

\section{Problem 4}

Proof. Suppose $x \in C$. We can conclude that $x \notin A \setminus
B$. From, $x \notin A \setminus B$, it follows that $x \in B$.
Therefore, if $x \in C$, then $x \in B$.

\section{Problem 5}

Proof. Suppose $a \in A \setminus B$. We will prove the
contrapositive. It follows that $a \in A$ from our assumption. Also,
$a \in C$ from which we can conclude that $a \in B$ since $A \cap C
\subseteq B$ But this contradicts the fact that $a \notin B$.
Therefore $a \notin A \setminus B$.

\section{Problem 6}

Proof. Suppose $a \notin C$. We will prove the contrapositive. From $a
\notin B \setminus C$, we can conclude that either $a \notin B$ or $a
\in C$ is true. Since $a \in A$, we can conclude that $a \in B$ from
$A \subseteq B$. So, $a \notin B$ is false. Therefore, $a \in C$
should be true. But this contradicts the fact that $a \notin C$.
Therefore $a \in C$.

\section{Problem 7}

Proof. Suppose $y = 0$. We will prove the contrapositive. Solving $y +
x = 2y - x$ by substituting $y=0$ we get that $x=0$. But it
contradicts the fact that both $x$ and $y$ cannot be $0$. Therefore,
$y \neq 0$.

\section{Problem 8}

Proof. Suppose $a < 1/a < b < 1/b$. Multiplying  the inequality by
$1/a < 1/b$ by $ab$ we should get $a < b$. From this, we can conclude
that $ab$ is a negative number. Since $a < b$, we can conclude that
$a$ is the negative number. Multiplying inequality $a < 1/a$ by
negative number we get $a^2 < 1$. From that we can conclude that $a <
-1$. Therefore, if $a < 1/a < b < 1/b$, then $a < -1$.

\section{Problem 9}

Proof. Suppose $x^2y = 2x + y$. Suppose $y \neq 0$ Suppose $x = 0$.
Substituting this into the equivalent $x^2y = 2x + y$, we get $y = 0$.
But this contradicts the fact that $y \neq 0$. Therefore $x \neq 0$.
Thus, if $y \neq 0$, then $x \neq 0$.

\section{Problem 10}

Proof. Suppose $x \neq 0$. Suppose $y = 3x^2 + 2y / x^2 + 2$. Solving
it we get $x^2(3 - y) = 0$. Since $x \neq 0$, we can conclude that $y
= 3$.

\section{Problem 11}

Solution (a)

If conclusion of the theorem is false, then $\neg( x \neq 3 \land y
\neq 8)$. That is equivalent to $x = 3 \lor y = 8$.

Solution (b)

First example: $x = 3$ and $y = 7$.
Second example: $x = 2$ and $y = 8$.

\section{Problem 12}

Solution (a)

This is plain wrong: Since $x \notin B$ and $B \subseteq C$, $x \notin
C$.

From $x \notin B$ and $B \subseteq C$, we cannot conclude that $x
\notin C$.

Solution (b)

$ C = \{1,2,3,4,5\} $
$ A = \{1,2\}$
$ B = \{3,4,5\}$
$ x = 1 $
\end{document}

