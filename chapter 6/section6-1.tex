%%%%%%%%%%%%%%%%%%%%%%%%%%%%%%%%%%%%%%%%%
% Author: Sibi <sibi@psibi.in>
%%%%%%%%%%%%%%%%%%%%%%%%%%%%%%%%%%%%%%%%%
\documentclass{article}
\usepackage{graphicx}
\usepackage{verbatim}
\usepackage{amsmath}
\usepackage{amsfonts}
\usepackage{amssymb}
\usepackage{tabularx}
\usepackage{mathtools}
\newcommand{\BigO}[1]{\ensuremath{\operatorname{O}\bigl(#1\bigr)}}
\setlength\parskip{\baselineskip}
\begin{document}
\title{Chapter 6 (Section 6.1)}
\author{Sibi}
\date{\today}
\maketitle

% See here: http://tex.stackexchange.com/a/43009/69223
\DeclarePairedDelimiter\abs{\lvert}{\rvert}%
\DeclarePairedDelimiter\norm{\lVert}{\rVert}%

% Swap the definition of \abs* and \norm*, so that \abs
% and \norm resizes the size of the brackets, and the 
% starred version does not.
\makeatletter
\let\oldabs\abs
\def\abs{\@ifstar{\oldabs}{\oldabs*}}
%
\let\oldnorm\norm
\def\norm{\@ifstar{\oldnorm}{\oldnorm*}}
\makeatother
\newpage

\section{Solution 1}
By mathematical induction, \\
Base case. When $n = 0$, $0 = \frac{n(n+1)}{2} = 0$.
Induction step. Let $n$ be an arbitrary element and suppose $0 + 1 + 2
+ ... + n = \frac{n(n+1)}{2}$. Then,
\begin{align*}
  0 + 1 + 2 + ... + n + (n + 1) = \frac{n(n+1)}{2} + (n + 1) \\
  = (n+1)(\frac{n}{2} + 1) \\
  = \frac{(n+1)(n+2)}{2}
\end{align*}

\section{Solution 2}
By mathematical induction,

Base case. When $n = 0$, both sides of the equation becomes $0$.

Induction step. Let $n$ be arbitrary and suppose $0^2 + 1^2 + 2^2 +
... + n^2 = \frac{n(n+1)(2n+1)}{6}$. Then,
\begin{align*}
  0^2 + 1^2 + 2^2 + ... + n^2 + (n+1)^2 = \frac{n(n+1)(n+2)}{6} + (n +
  1)^2 \\
  = (n+1)(\frac{n(2n + 1)}{6} + (n + 1)) \\
  = (n+1)(\frac{2n^2 + n + 6n + 6}{6}) \\
  = \frac{(n+1)(n+2)(n+3)}{6} \\
\end{align*}
\section{Solution 3}
By mathematical induction,

Base case. When $n=0$, both sides of the equation becomes $0$.

Induction step. Let $n$ be an arbitrary element in $N$. Suppose $0^3 +
1^3 + 2^3 + ... + n^3 = [\frac{n(n+1)}{2}]^2$. Then,

\begin{align*}
  0^3 + 1^3 + 2^3 + ... + n^3 + (n+1)^3= [\frac{n(n+1)}{2}]^2 + (n+1)^3 \\
  = (n+1)^2(\frac{n^2 + 4n + 4}{4}) \\
  = \frac{(n+1)^2(n+2)^2}{4} \\
  = [\frac{(n+1)(n+2)}{2}]^2 
\end{align*}

\section{Solution 4}
By mathematical induction,

Base case. When $n=1$, both sides of the equation becomes 1.

Induction step. Let $n$ be an arbitrary element and suppose $1+3+5+...
+ (2n-1) = n^2$. Then,

\begin{align*}
  1 + 3 + 5 + ... + (2n - 1) = n^2 \\
  1 + 3 + 5 + ... + (2n - 1) + (n+1)^2 = n^2 + (n+1)^2 \\
  1 + 3 + 5 + ... + (2n - 1) + n^2 + 1 + 2n = n^2 + (n+1)^2 \\
  1 + 3 + 5 + ... + (2n + 1) = (n+1)^2.
\end{align*}

\section{Solution 5}
By mathematical induction,

Base case. When $n=0$, both sides of the equation becomes $0$.

Inductive step. Let $n$ be an arbitrary element and suppose $0.1 + 1.2
+ 2.3 + ... + n(n+1) = \frac{n(n+1)(n+2)}{3}$. Then,

\begin{align*}
  0.1 + 1.2 + 2.3 + ... + n(n+1) + (n+1)(n+2) = \frac{n(n+1)(n+2)}{3}
  + (n+1)(n+2) \\
  = (n+1)(n+2)(\frac{n}{3} + 1) \\
  = \frac{(n+1)(n+2)(n+3)}{3}
\end{align*}
\section{Solution 6}
Guess: $\frac{n(n+1)(n+2)(n+3)}{4}$

By mathematical induction,

Base case. When $n=0$, both sides of the equation becomes $0$.

Induction step. Let $n$ be an arbitrary element. Suppose $0.1.2 +
1.2.3 + 2.3.4 + ... + n(n+1)(n+2) = \frac{n(n+1)(n+2)(n+3)}{4}$. Then,

\begin{align*}
  0.1.2 + 1.2.3 + 2.3.4 + ... + n(n+1)(n+2) + (n+1)(n+2)(n+3) =
  \frac{n(n+1)(n+2)(n+3)}{4} +  (n+1)(n+2)(n+3) \\
  = (n+1)(n+2)(n+3)(\frac{n}{4} + 1) \\
  = \frac{(n+1)(n+2)(n+3)(n+4)}{4}
\end{align*}

\section{Solution 7}
Guess: $\frac{3^{n+1} - 1}{2}$

By mathematical induction,

Base case. When $n=0$, both sides of the equation becomes 1.

Induction case. Let $n$ be an arbitrary element. Suppose $3^0 + 3^1 +
3^2 + ... + 3^n = \frac{3^{n+1} - 1}{2}$. Then,

\begin{align*}
  3^0 + 3^1 + 3^2 + ... + 3^n + 3^{n+1} = \frac{3^{n+1} - 1}{2} +
  3^{n+1} \\
  = \frac{3^{n+1} - 1 + 2.3^{n+1}}{2} \\
  = \frac{3^{n+2} - 1}{2}
\end{align*}

\section{Solution 8}
By mathematical induction,

Base case. When $n = 1$, both sides of the equation becomes $1/2$.

Induction step. Let $n$ be an arbitrary element such that $n >= 1$ and
suppose $1 - \frac{1}{2} + \frac{1}{3} + ... + \frac{1}{2n - 1} -
  \frac{1}{2n} = \frac{1}{n + 1} + \frac{1}{n+2} + \frac{1}{n+3} + ...
  + \frac{1}{2n} $. Then,
  \begin{align*}
    1 - \frac{1}{2} + \frac{1}{3} + ... + \frac{1}{2n - 1} -
    \frac{1}{2n} + \frac{1}{2n+1} - \frac{1}{2n+2} \\
    = \frac{1}{n+1} + \frac{1}{n+2} + \frac{1}{n+3} + ... +
    \frac{1}{2n} + \frac{1}{2n+1} - \frac{1}{2n+2} \\
    = \frac{1}{n+2} + \frac{1}{n+3} + ... + \frac{1}{2n} +
    \frac{1}{2n+1} + \frac{1}{2n+2}
  \end{align*}

  as required.

\section{Solution 9}
\subsection{Solution (a)}
By mathematical induction,

Base case. When $n = 0$, $\exists k \in Z$ such that $2k = n^2 + 2$.
Here if $k = 0$, then base case holds.

Induction step. Let $n$ be an arbitrary element in $N$. Suppose $2k =
n^2 + n$ for some $k \in Z$. Thus,
\begin{align*}
  (n + 1)^2 (n + 1) = n^2 + 1 + 2n + n + 1 \\
  = n^2 + 3n + 2 \\
  = n^2 + n + 2n + 2 \\
  = 2k + 2n + 2 \\
  = 2(k + n + 2)
\end{align*}

Therefore $2 \mid (n+1)^2 + (n+1)$ as required.

\subsection{Solution (b)}
By mathematical induction,

Base case. When $ n = 0$, it holds that $6 \mid (n^3 - n)$ for $0 \in
Z$.

Induction step. Let $n$ be an arbitrary element in $N$ such that $6
\mid (n^3 - n)$.
\begin{align*}
  (n+1)^3 - (n+1) \\
  = (n+1)((n+1)^2 - 1) \\
  = (n+1)(n^2 + 1 + 2n - 1) \\
  = (n+1)(n^2 + 2n) \\
  = n^3 + 3n^2 + 2n \\
  = n^3 - n + 3n^2 + 2n \\
  = 6k + 3n^2 + 3n \\
  = 6(k + n^2/2 + n/2) \\
  = 6(k + n(n+1)/2)
\end{align*}
Now $n(n+1)/2$ is the sum of number series upto $n$. So, $6 \mid
(n+1)^3 - (n+1)$ holds.

\section{Solution 10}
By mathematical induction,

Base case. When $n = 0$, $64 \mid (9^n - 8n - 1)$ holds for $0$.

Induction step. Let $n$ be an arbitrary element in $N$ such that $64k
= (9^n - 8n - 1)$. Then,
\begin{align*}
  9^{n+1} - 8(n+1) - 1 = 9^{n+1} - 8n - 9 \\
  = 9^{n+1} - 9n - 9 + n \\
  = 9(9^n - n - 1) + n \\
  = 9(9^n - 8n - 1 + 7n) + n \\
  = 9(64k + 7n) + n \\
  = 64(9k + n) \\
\end{align*}
So, $64 \mid (9^{n+1} - 8(n+1) - 1)$ as required.

\section{Solution 11}
By mathematical induction,

Base case. When $n = 0$, $9 \mid (4^n + 6n - 1)$ holds for $0$.

Induction step. Let $n$ be an arbitrary element in $N$ and suppose $9k
= 4^n + 6n - 1$. Thus,
\begin{align*}
  4^{n+1} + 6(n+1) - 1 = 4^{n+1} + 6n + 5 \\
  = 4^{n+1} + 4n + 2n + 5 \\
  = 4(4^{n} + n) + 2n + 5 \\
  = 4(4^{n} + 6n - 1 - 5n + 1) + 2n + 5 \\
  = 4(9k - 5n + 1) + 2n + 5 \\
  = 4.9k - 18n + 9 \\
  = 9(4k - 2n + 1)
\end{align*}
So, $9 \mid 4^{n+1} + 6(n+1) - 1$ as required.

\section{Solution 12}
Let $a$ and $b$ be arbitrary integer. By mathematical induction,

Base case. When $n = 0$, $(a-b) \mid (a^n - b^n)$ holds for $0$.
Induction step. Let $n$ be arbitrary element in $N$ and suppose
$(a-b)k = a^n - b^n$ . Thus,
\begin{align*}
  a^{n+1} - b^{n+1} = a(a^n - b^n) + b^n(a-b) \\
  = a(a-b)k + b^n(a-b) \\
  = (a-b)(ak + b^n) \\
\end{align*}
So, $(a-b) \mid (a^{n+1} - b^{n+1})$ as required.

\section{Solution 13}
Let $a$ and $b$ be arbitrary element. By mathematical induction,

Base case. When $n=0$, $(a+b) \mid (a^{2n+1} + b^{2n + 1})$ holds for
$1 \in Z$.

Induction step. Let $n$ be an arbitrary element in $N$ such that
$(a+b)k = a^{2n+1} + b^{2n+1}$. Thus,
\begin{align*}
  a^{2(n+1) + 1} + b^{2(n+1) + 1} = a^{2n + 3} + b^{2n + 3} \\
  = a^2(a^{2n + 1}+ b^{2n+1}) + b^{2n+1}(b^2 - a^2) \\
  = a^2(a+b)k + b^{2n+1}(b-a)(b+a) \\
  = (a+b)(a^2k + b^{2n+1}(b-a))
\end{align*}
So, $(a+b) \mid a^{2(n+1) + 1} + b^{2(n+1) + 1}$ as required.

\section{Solution 14}
By mathematical induction,

Base case. When $n = 10$, $2^n > n^3$ holds since $2^{10} > 2^3$.

Induction step. Let $n$ be an arbitrary element and suppose $n >= 10$
and $2^n > n^3$. Then,
\begin{align*}
  2^{n+1} = 2.2^n \\
  > 2.n^3 (\text{Inductive hypothesis}) \\
  = n^3 + n^3 \\
  >= n^3 + 10n^2 (\text{Since n >= 10}) \\
  = n^3 + 3n^2 + 7n^2 \\
  >= n^3 + 3n^2 + 70n (\text{Since n >= 10}) \\
  = n^3 + 3n^2 + 3n + 67n \\
  > n^3 + 3n^2 + 3n + 1 \\
  > (n+1)^3
\end{align*}
So, $2^{n+1} > (n+1)^3$ as required.

\section{Solution 15}
By mathematical induction,

Base case. When $n=0$, $n \equiv 0 (mod 3)$ holds for $0 \in Z$.

Induction step. Let $n$ be an arbitrary element in $N$. Suppose
$n \equiv 0 (\text{mod } 3)$ or $n \equiv 1 (\text{mod } 3)$ or
$n \equiv 2 (\text{mod } 3)$.

Case 1. $n \equiv 0 (\text{mod } 3)$. It follows that $n = 3a$ for
some $a \in Z$. Adding $1$ on both sides, we get $n+1 = 3a + 1$. So,
$(n+1) - 1 = 3a$. Therefore, $n+1 = 1(\text{mod } 3)$. So, either
$n+1 = 0(\text{mod } 3)$ or $n+1 = 1(\text{mod } 3)$ or
$n+1 = 2(\text{mod } 3)$.

Case 2. $n \equiv 1 (\text{mod } 3)$. It follows that $n-1=3b$ for
some $b \in Z$. We can rewrite it as $(n+1) - 2 = 3b$. So,
$(n+1) = 2(\text{mod } 3)$. So, $(n+1) \equiv 2(\text{mod } 3)$. So,
either $n+1 = 0(\text{mod } 3)$ or $n+1 = 1(\text{mod } 3)$ or
$n+1 = 2(\text{mod } 3)$.

Case 3. $n \equiv 2 (\text{mod } 3)$. It follows that $n-2 = 3c$ for
some $c \in Z$. Rewriting it, we get $n+1 = 3(c+1)$. So,
$n+1 \equiv 0 (\text{mod } 3)$. So, either $n+1 = 0(\text{mod } 3)$ or
$n+1 = 1(\text{mod } 3)$ or $n+1 = 2(\text{mod } 3)$.

\section{Solution 16}
By mathematical induction,

Base case. When $n = 1$, both sides of the equation becomes $8$.

Induction step. Let $n >=1$ be an arbitrary element. Suppose $2.2^1 +
3.2^2 + 4.2^3 + ... + (n+1)2^n = n2^{n+1}$. Then,
\begin{align*}
  2.2^1 + 3.2^2 + 4.2^3 + ... + (n+2)2^{n+1} \\
  = (2.2^1 + 3.2^2 + ... + (n+1)2^n) + (n+2)2^{n+1} \\
  = n2^{n + 1} + (n+2)2^{n+1} \\
  = 2^{n+1}(n + n + 2) \\
  = 2^{n+2}(n+1)
\end{align*}

\end{document}
