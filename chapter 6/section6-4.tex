%%%%%%%%%%%%%%%%%%%%%%%%%%%%%%%%%%%%%%%%%
% Author: Sibi <sibi@psibi.in>
%%%%%%%%%%%%%%%%%%%%%%%%%%%%%%%%%%%%%%%%%
\documentclass{article}
\usepackage{graphicx}
\usepackage{verbatim}
\usepackage{amsmath}
\usepackage{amsfonts}
\usepackage{amssymb}
\usepackage{tabularx}
\usepackage{mathtools}
\newcommand{\BigO}[1]{\ensuremath{\operatorname{O}\bigl(#1\bigr)}}
\setlength\parskip{\baselineskip}
\begin{document}
\title{Chapter 6 (Section 6.2)}
\author{Sibi}
\date{\today}
\maketitle

% See here: http://tex.stackexchange.com/a/43009/69223
\DeclarePairedDelimiter\abs{\lvert}{\rvert}%
\DeclarePairedDelimiter\norm{\lVert}{\rVert}%

% Swap the definition of \abs* and \norm*, so that \abs
% and \norm resizes the size of the brackets, and the 
% starred version does not.
\makeatletter
\let\oldabs\abs
\def\abs{\@ifstar{\oldabs}{\oldabs*}}
%
\let\oldnorm\norm
\def\norm{\@ifstar{\oldnorm}{\oldnorm*}}
\makeatother
\newpage

\section{Solution 1}

\subsection{Solution (a)}

($\Rightarrow$) Suppose $\forall n Q(n)$. Let $n$ be an arbitrary element. Since $Q(n+1)$ is true, it follows $\forall m < n + 1(P(m))$. Let $m = n$. Since $n < n + 1$, it follows that $P(n)$. Since $n$ was arbitrary it follows that $\forall n P(n)$.

($\Leftarrow$) Suppose $\forall n P(n)$. Then it follows that $\forall k < n P(k)$. So, $Q(n)$.

\subsection{Solution (b)}

By mathematical induction on $n$,

Base case. $n = 0$. $\forall k < n P(k)$ is vacously true.
Induction step. Suppose $Q(n)$. So, $\forall k < n P(k)$. Since $k < n + 1$, it follows that $\forall k < n + 1 P(k)$. Now $\forall k < n + 1 P(k)$ is equivalent to $Q(n + 1)$.

\section{Solution 2}
$P(q) = \forall q \in N (q > 0 \implies \neg \exists p \in N(p/q = \sqrt{2}))$

Let $q$ be an arbitrary element in $N$. Suppose $\forall k < q(P(k))$. Suppose $q > 0$. Let us prove by contradiction. There exists some $p \in N$ such that $p/q = \sqrt{2}$. So, $p^2 = 2q^2$. It follows that $p$ and $q$ are even. So, $p = 2p'$ and $q = 2q'$. Now, $p'/q' = \sqrt{2}$. Since $q' < q$, it contradicts with the Inductive hypothesis. So, $\sqrt{2}$ is irrational.

\section{Solution 3}
\subsection{Solution (a)}
Suppose $\sqrt{6}$ is rational. Then, $\exists p \in Z^{+} \exists q \in Z^{+}(p/q = \sqrt{6})$. So the set $S = \{q \in Z^{+} \mid \exists p \in Z^{+}(p/q = \sqrt{6})\}$ is non-empty. By well ordering principle, let $q$ be the smallest element in $S$. Since $q \in S$, we can choose some $p \in Z^{+}$ such that $p/q = \sqrt{6}$. So, $p^2 = 6q^2$. We know that both $p$ and $q$ are even. So, let $p' = 2p$ and $q' = 2q$ such that $p'/q' = \sqrt{6}$. Since $q' < q$ it contradicts that $q$ is the smallest element of the set. So, $\sqrt{6}$ is irrational.

\subsection{Solution (b)}
Suppose $\sqrt{2} + \sqrt{3} = p/q$ where $p \in Z^{+}$ and $q \in Z^{+}$. So, $p^2 / q^2 = 5 + 2\sqrt{6}$. So, $\sqrt{6} = (p^2 - 5q^2)/q^2$. This contradicts the part (a).

\section{Solution 4}
To prove: $\forall n >= 12 \exists a \in \mathbb{N} \exists b \in \mathbb{N}(3a + 7b = n)$.

We have to prove it like this:

\begin{itemize}
  
\item 
  Prove for n=12,13,14
  
\item 
  $\forall n >= 15 \exists a \exists b. (n = 3a + 7b)$
\end{itemize}

I will just prove the second item here. Let $n >= 15$ be an arbitrary element.Suppose  $\forall k(  15 <= k < n)$, there exists $a$ and $b$ such that $k = 3a + 7b$. Since $n - 3 < n$, it follows that $n - 3 = 3a + 7b$ from inductive hypothesis. So, $n = 3(a + 1) + 7b$. Let $a + 1 = a'$, So, $\exists a' \exists b (n = 3a' + 7b)$. Since $n$ was arbitrary, it follows that $\forall n >= 15 \exists a \exists b. (n = 3a + 7b)$.

Also see this: http://math.stackexchange.com/a/1745857/124772

\end{document}
