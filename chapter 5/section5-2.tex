%%%%%%%%%%%%%%%%%%%%%%%%%%%%%%%%%%%%%%%%%
% Author: Sibi <sibi@psibi.in>
%%%%%%%%%%%%%%%%%%%%%%%%%%%%%%%%%%%%%%%%%
\documentclass{article}
\usepackage{graphicx}
\usepackage{verbatim}
\usepackage{amsmath}
\usepackage{amsfonts}
\usepackage{amssymb}
\usepackage{tabularx}
\usepackage{mathtools}
\newcommand{\BigO}[1]{\ensuremath{\operatorname{O}\bigl(#1\bigr)}}
\setlength\parskip{\baselineskip}
\begin{document}
\title{Chapter 5 (Section 5.2)}
\author{Sibi}
\date{\today}
\maketitle

% See here: http://tex.stackexchange.com/a/43009/69223
\DeclarePairedDelimiter\abs{\lvert}{\rvert}%
\DeclarePairedDelimiter\norm{\lVert}{\rVert}%

% Swap the definition of \abs* and \norm*, so that \abs
% and \norm resizes the size of the brackets, and the 
% starred version does not.
\makeatletter
\let\oldabs\abs
\def\abs{\@ifstar{\oldabs}{\oldabs*}}
%
\let\oldnorm\norm
\def\norm{\@ifstar{\oldnorm}{\oldnorm*}}
\makeatother
\newpage

\section{Solution 1}
One-to-one: (c) \\
Onto: (a)

\section{Solution 2}
One to one: (c) \\
Onto: (b)-function g, (c)

\section{Solution 3}
One to one: No one \\
Onto: (a), (b), (c)

\section{Solution 4}
One to one: (a), (b), (c) \\
Onto: (b), (c)

\section{Solution 5}
\subsection{Solution 5 (a)}
Let $a_1$ and $a_2$ be arbitrary element in $A$. Suppose $f(a_1) =
f(a_2)$. It follows that $(a_1 + 1)(a_2 - 1) = (a_2 + 1)(a_1 - 1)$.
Solving it, we get $a_1 = a_2$. Since $a_1$ and $a_2$ are arbitrary we
can conclude that $f$ is one to one.

Let $a_1$ be arbitrary element in $A$. Let $a_2 = a_1 + 1 / a_1 - 1$.
Then it follows that $f(a_2) = f(a_1)$. So, $\forall a_1 \in A \exists
a_2 \in A(f(a_2) = f(a_1))$.

\subsection{Solution 5 (b)}
$(\Rightarrow)$ Suppose $(a_1, a_2) \in f \circ f$. Then it follows
$\exists a_3 \in A (a_1, a_2) \in f$ and $(a_3,a_2) \in f$. We know
that $f(x) = \frac{x + 1}{x - 1}$. So, $a_3 = \frac{a_1 + 1}{a_1 -
  1}$. Solving it, we get $a_1 = \frac{a_3 + 1}{a_1 - 1}$. Similarly
from $(a_3, a_2)$ we get, $a_2 = \frac{a_3 + 1}{a_1 - 1}$. So, $a_1 =
a_2$. Since $a_1 = a_2$, it follows that $(a_1, a_2) \in i_A$. So, $f
\circ f \subseteq i_A$.

$(\Leftarrow)$ Suppose $(a_1, a_2) \in i_A$. Then it follows that $a_1
= a_2$. Since $a_1 \in A$, it follows that $(a_1, \frac{a_1 + 1}{a_1 -
  1}) \in f$. Deducing, we get $(\frac{a_1 + 1}{a_1 - 1}, a_1) \in f$.
So, $(a_1, a_2) \in f \circ f$. Since $a_1$ and $a_2$ are arbitrary,
it follows that $i_A \subseteq f \circ f$.

\section{Solution 6}
\subsection{Solution 6 (a)}
$f(2) = \{y \in R \mid y^2 < 2\}$

\subsection{Solution 6 (b)}
\subsubsection{Proof for one-to-one}
Let $r_1, r_2$ be arbitrary element in $R$ such that $f(r_1) =
f(r_2)$. So, $\{y \in R \mid y^2 < r_1\} = \{y \in R \mid y^2 < r_2\}$
. Now those two sets can be equivalent only if $r_1 = r_2$
\subsubsection{Proof that it is not onto}
Suppose $a = \{1,2\}$. such that $f(r) = a$ for some $r$. It follows
that $\{y \in R \mid y^2 < r\} = \{1,2\}$. But for any $x \in R$,
$f(x) \neq \{1,2\}$.

\section{Solution 7}
\subsection{Solution 7(a)}
$\{1,2,3,4\}$

\subsection{Solution 7(b)}
\subsubsection{Proof that it is not one to one}
Suppose $b_1 = \{\{1,2\}, \{3,4\}\}$ and $b_2 = \{\{1,2,3\}, \{4\}\}$.
It follows that $f(b_1) = f(b_2)$. But $b_1 \neq b_2$. So, $f$ is not
one to one.

\subsubsection{Proof for onto}
Let $a$ be an arbitrary element in $A$. Suppose $b = \{\{a\}\}$. Then
$f(b) = \cup{b} = a$. Since $a$ was arbitrary, we can conclude that
$f$ is onto.

\section{Solution 8}
\subsection{Solution 8 (a)}
Suppose $g \circ f$ is onto. Then it follows that $\forall c \in C
\exists a \in A(g \circ f(a) = c)$. Let $c_1$ be arbitrary element in
$C$. From universal instantiation, it follows that $\exists a \in A(g
\circ f(a) = c_1)$. So there is $\exists a \in A$ such that $g \circ
f(a) = c_1$. Since $f$ is a function, let $f(a) = b$ for some $b \in
B$. Then it follows that $g(b) = c_1$.

\subsection{Solution 8 (b)}
Suppose $g \circ f$ is one to one. Let $a_1, a_2$ be arbitrary element
of $A$. Suppose $f(a_1) = f(a_2)$. We know that $g \circ f(a_1) = g
\circ f(a_2)$, So, $a_1 = a_2$. Since $a_1$ and $a_2$ are arbitrary,
we can conclude that $f$ is one-to-one.

\section{Solution 9}
\subsection{Solution 9 (a)}
Suppose $f$ is onto and $g$ is not one to one. Suppose $a_1, a_2$ be
some element in $A$. Let $f(a_1) = b_1$ and $f(a_2) = b_2$. We know
that $g$ is not one to one, so $g(b_1) = g(b_2) \land b_1 \neq b_2$.
From $g(b_1) = g(b_2)$, it follows that $g \circ f(a_1) = g \circ
f(a_2)$. Similarly we know that $b_1 \neq b_2$, so $a_1 \neq a_2$. So,
$\exists a_1 \in A \exists a_2 \in A(g \circ f(a_1) = g \circ f(a_2)
\land a_1 \neq a_2$. Hence $g \circ f$ is not one to one.

\subsection{Solution 9 (b)}
Suppose $f$ is not onto and $g$ is one to one. Since $f$ is not onto,
it follows that $\exists b \in B \forall a \in A(f(a) \neq b)$. Since
$g$ is one to one, it follows that $\neg b_1 \in B \exists b \in
B(f(b_1) = f(b) \land b_1 \neq b)$. Suppose $g(b) = c$. From $\exists
b \in B \forall a \in A((f(a) \neq b)$, it follows that $g \circ f(a)
\neq c$ and since $g$ one to one, $\neg b_2 \in B(g(b_2) = c \land b^2
\neq b)$. So, $g \circ f$ is not onto.

\section{Solution 10}
\subsection{Solution (a)}
Suppose $f$ is one to one. Let $c_1$ and $c_2$ be arbitrary element in
$C$. Since $c_1 \in C$ and $C \subseteq A$ it follows that $f(c_1)
\neq fc(c_2) \lor c_1 = c_2$. So, $f \mid C$ is one to one.

\subsection{Solution (b)}
Suppose $f \mid C$ is onto. Let $a$ be arbitrary element in $A$. Let
us consider the cases:
\\ Case 1. $a \in C$. We know that $f \mid C$ is onto. So, since $C
\subseteq A$, we can conclude that $f$ is onto.
\\ Case 2. $a \notin C$. Suppose $f(a) = b$ for some $b \in B$. We
already know that $f \mid C :: C -> B$ is onto. So, for some $a_1 \in
A$, $f \mid C (a_1) = b$. So, $\forall a \in A \exists b \in B(f(a) =
b)$.

\subsection{Solution (c)}
\subsubsection{Example for (a)}
\begin{align*}
  A = \{1,2\}
  B = \{3\} \\
  C = \{1\} \\
  f = \{(1,3), (2,3)\} \\
  f \mid C = \{(1,3)\} 
\end{align*}

\subsubsection{Example for (b)}
\begin{align*}
  A = \{1,2\}
  B = \{3,4\} \\
  C = \{1\} \\
  f = \{(1,3), (2,4)\} \\
  f \mid C = \{(1,3)\} 
\end{align*}

\section{Solution 11}
\subsection{Solution (a)}
Suppose $A$ has more than one element. Let $a_1 \in A$ and $a_2 \in A$
such that $a_1 \neq a_2$. Since $f$ is a constant function $\exists b
\in B, f(a_1) = b$ and $f(a_2) = b$. So, $\exists a_1 \in A \exists
a_2 \in A(f(a_1) = f(a_2) \land a_1 \neq a_2)$. So, $f$ is not onto.

\subsection{Solution (b)}
Suppose $B$ has more than one element. Let $b_1 \in B$ and $b_2 \in B$
such that $b_1 \neq b_2$. Since $f$ is a constant either $f(a) = b_1$
or $f(a) = b_2$ where $a$ is an arbitrary number in $A$. Let us
consider the cases:
Case 1. $f(a) = b_2$. For $b_1 \in B$. $\forall a \in A(f(a) \neq
b_1)$. So, $f$ is not onto.
Case 2. $f(a) = b_1$. For $b_2 \in B$, $\forall a \in A(f(a) \neq
b_2)$. So, $f$ is not onto.

\section{Solution 12}
$(\Rightarrow)$ Suppose $f \cup g$ is one to one. Let $c$ be an
arbitrary element in $Ran(f)$. Then $\exists a \in A$ such that $f(a)
= c$. We have to prove that $c \notin Ran(g)$. We know that $(a,c) \in
f \cup g$. Since $f \cup g$ is one to one, $\neg \exists a_1 \in A \cup B \exists
a \in A \cup B(f \cup g(a_1) = f \cup g(a) \land a_1 \neq a)$. We know
that $f$ already has $c$ in it's Range. So, $\neg \exists b \in B$,
such that $g(b) = c$. So, $c \notin Ran(R)$.

$(\Leftarrow)$ Suppose $Ran(f) \cap Ran(g) = \emptyset$. Let $x_1,
x_2$ be arbitrary element in $A \cup B$ such that $f \cup g(x_1) = f
\cup g(x_2)$. We know that $A$ and $B$ are disjoint. So, let us
consider the cases:
Case 1. $x_1 \in A$ Then it follows that $f(x_1) = c$ for some $c \in
C$. So, $f \cup g(x_1) = c$. Also, $f \cup g(x_2) = c$. Now let us
consider the case for $x_2$. \\
$x_1 \in A$ Then it follows that $f(x_2) = c$. But we know that $f$ is
one to one. So this is only possible if $x_1 = x_2$. \\
$x_2 \in B$. Then it follows that $f \cup g(x_2) = c$. So, $(x_2, c)
\in g$. But we know that $Ran(f) \cap Ran(g) = \emptyset$. So this is
not possible. \\
Case 2. Same as case one.

\section{Solution 13}
Suppose $S$ is one to one. Let $a$ be an arbitrary element in $A$. \\
Existence proof. Since $S \circ R$ is a function, $\exists c \in C (S
\circ R(a) = c$. So, $\exists b \in B$ such that $(a,b) \in R$. \\
Uniqueness proof. Suppose $(a,b_1) \in R$ and $(a,b_2) \in R$. From $S
\circ R$, it follows that $(b_1, c_1) \in S$ and $(b_2, c_2) \in S$.
So, $(a,c_1) \in S \circ R$ and $(a,c_2) \in S \circ R$. Since $S
\circ R$ is a function, it follows that $c_1 = c_2$. So, $(b_1, c_1)
\in S$ and $(b_2, c_1) \in S$. But since $S$ is one to one this is not
possible. So, $b_1 = b_2$.

\section{Solution 14}
\subsection{Solution (a)}
Suppose $R$ is reflexive and $f$ is onto. Let $b$ be arbitrary element
in $B$. Since $f$ is onto, it follows that $\exists a \in A$ such that
$f(a) = b$. Now $a \in A$ and since $R$ is reflexive $(a,a) \in R$.
From the definition of $S$, it follows that $(b,b) \in S$. Since $b$
is arbitrary we can conclude that $S$ is reflexive.

\subsection{Solution (b)}
Suppose $R$ is transitive and $f$ is one to one. Let $b_1, b_2$ and
$b_3$ be arbitrary element in $B$ such that $(b_1, b_2) \in S$ and
$(b_2, b_3) \in S$. From $(b_1, b_2) \in S$ it follows that $(a_1,
a_2) \in R$ such that $f(a_1) = b_1$ and $f(a_2) = b_2$. Similarly,
$(a_3,a_4) \in R$ such that $f(a_3) = b_2$ and $f(a_4) = b_3$. But
$f(a_2) = b_2$ and $f(a_3) = b_2$ is not possible since $f$ is one to
one. So, $a_2 = a_3$. So from $(a_1, a_2) \in R$ and $(a_2, a_4) \in
R$ it follows that $(a_1, a_4) \in R$. From the definition of $S$, it
follows that $(b_1, b_3) \in S$. Since $b_1, b_2$ and $b_3$ are
arbitrary we can conclude that $S$ is transitive.

\section{Solution 15}
\subsection{Solution (a)}Let $X$ be an arbitrary element in $A/R$.
From $X \in A / R$, it follows that $\exists x \in A$ such that $x \in
X$. So, from the definition of $g$, $g(x) = X$. Since $X$ is
arbitrary, we can conclude that $g$ is onto.

\subsection{Solution (b)}
$(\Rightarrow)$ Suppose $g$ is one to one. Let $a_1, a_2$ be arbitrary
element on $A$ such that $(a_1, a_2) \in R$. Since $R$ is reflexive it
follows that $(a_1, a_2) \in R$. From the definition of $g$, it
follows that $g(a_1) = [a_1]_R$. From $(a_1, a_2) \in R$, it follows
that $[a_1]_R = [a_2]_R$. So, $g(a_1) = [a_1]_R$ and $g(a_2) =
[a_1]_R$. But we know that $g$ is one to one, so $a_1 = a_2$. So,
$(a_1, a_2) \in i_A$. Since $a_1$ and $a_2$ are arbitrary element, $R
\subseteq i_A$. Let $a_1, a_2$ be arbitrary element on $i_A$ such that
$(a_1, a_2) \in i_A$. From the property of identity relation, it
follows that $a_1 = a_2$. Since $R$ is reflexive, it follows that
$(a_1, a_2) \in R$. So, $i_A \subseteq R$.

$(\Leftarrow)$ Suppose $R = i_A$. Let $(a_1, a_2)$ be arbitrary
element on $A$ such that $g(a_1) = g(a_2)$. From $g(a_1) = g(a_2)$, it
follows that $[a_1] = [a_2]$. So, $(a_1, a_2) \in R$. From $R = i_A$,
it follow that $a_1 = a_2$.

\section{Solution 16}
$(\Rightarrow)$ Suppose $h$ is one to one. Let $x,y$ be arbitrary
element in $A$ such that $f(x) = f(y)$. We know that $\forall x \in
A(h([x]_R) = f(x))$. So from $f(x) = f(y)$, it follows that $[x]_R =
[y]_R$. So, $(x,y) \in R$.

$(\Leftarrow)$ Suppose $\forall x \in A \forall y \in A(f(x) = f(y)
\implies xRy)$. Let $X_1, X_2$ be arbitrary element in $A/R$ such that
$h(X_1) = h(X_2)$. From $X_1 \in A/R$, it follows that $\exists a_1
\in A$ such that $a_1 \in X_1$. From $h([x]_R) = f(x)$, it follows
that $f(a_1) = f(a_2)$. From our initial assumption, we can conclude
that $a_1Ra_2$. So, $[a_1]_R = [a_2]_R$. So, $X_1 = X_2$. Since $X_1$
and $X_2$ are arbitrary, we can conclude that $h$ is one to one.

\section{Solution 17}
\subsection{Solution (a)}
Suppose $f$ is onto and $g:B \to C, h:B \to C \text{ dvand } g \circ f
= h \circ f$.
$(\Rightarrow)$ Let $b,c$ be arbitrary element in $B$ and $C$
respectively such that $(b,c) \in G$. Since $f$ is onto, $\exists a
\in A$ such that $f(a) = b$. From $g \circ f = h \circ f$, it follows
that $g(b) = h(b)$. From $(b,c) \in g$, it follows that $h(b) = c$.
So, $(b,c) \in h$. Since $b$ and $c$ are arbitrary we can conclude
that $g \subseteq h$.

\subsection{Solution (b)}
Let $b$ be arbitrary element in $B$. Suppose $C = \{c_1, c_2\}, g = B
\times \{c_1\}$ and $h=[B \setminus \{b\} \times \{c_1\} \cup
\{(b,c_2)\}]$. Then $g \neq h$. So by assumption $g \circ f \neq h
\circ f$. So, $g \circ f(a) = h \circ f(a)$ for $\exists a \in A$. By
the way $g$ and $h$ are defined, the only way they are not equal is
when $a=b$. So, $f(a) = b$. Since $b$ is arbitrary, $f$ is onto.

\section{Solution 18}
\subsection{Solution (a)}
Suppose $f$ is one to one, $g:A \to B, h: A \to B$ and $f \circ g = f
\circ h$.
$(\Rightarrow)$ Suppose $(a,b) \in g$. So, $a \in A$, $f \circ g(a) =
f(b)$. Suppose $f(b) = c$. From, $f \circ g = f \circ h$, it follows
that $f \circ h(a) = c$. Suppose $h(a) = b_1$. Then $f(b_1) = c$. But
since $f$ is one to one, so $b_1 = b$. So, $h(a) = b$. So, $g
\subseteq h$.
$(\Leftarrow)$. Similarly as above.

\subsection{Solution (b)}
Suppose $f$ is not one to one. We can prove by contradiction. $\exists
b_1 \in B \exists b_2 \in B$ such that $f(b_1) = f(b_2) \land b_1 \neq
b_2$. Let $g$ be defined like $\forall x \in A(g(x) = b_1)$.
Similarly, $h$ is $\forall x \in A(h(x) = b_2)$. Then it follows that
$g \neq h$. So from our assumption $f \circ g \neq f \circ h$. But we
know that $f \circ g = f \circ h$. So, $f$ is one to one.

\section{Solution 19}
\subsection{Solution (a)}
Let $a(x) = (x-1)^2 + 1$. it follows that $a : R \to R$ so $a \in F$.
We can prove that $h = a \circ f$, so $(h,f) \in R$. \\
We will prove by contradiction. Suppose there exists a function $a$
such that $g = a \circ f$. Or $x^3 + 1 = a(x^2 + 1)$. Solving it, we
get $a(x) = (\sqrt{x - 1})^3 + 1$. But this function is undefined for
$x - 1 < 1$. So, $(g,f) \notin R$.

\subsection{Solution (b)}
\subsubsection{Proof for Reflexive}
Let $f$ be an arbitrary element on $F$. We know that $id_R \in F$.
Since $f = id_R \circ f$, it follows that $(f,f) \in R$. Since $f$ is
arbitrary we can conclude that $R$ is reflexive.

\subsubsection{Proof for Transitive}
Let $f,g,h$ be an arbitrary element on $F$ such that $(f,g) \in R$ and
$(g,h) \in R$. From $(f,g) \in R$, it follows that $\exists h \in F$
such that $f = h \circ g$. Similarly $\exists a \in F(g = a \circ h)$.
So, $f = h \circ a \circ h$. So, $(f,h) \in R$. Since $f,g,h$ are
arbitrary, we can conclude that $R$ is transitive.

\subsection{Solution (c)}
Let $f$ be an arbitrary element in $F$. We know that $i_R \in F$.
Since $f = f \circ i_R$, it follows that $(f,i_R) \in R$. Since $f$ is
an arbitrary element $\forall f \in F(fRi_R)$

\subsection{Solution (d)}
Let $f$ be an arbitrary element in $F$.
($\Rightarrow$) Suppose $i_RRf$. Let $a_1, a_2$ be arbitrary element
in $R$ such that $f(a_1) = f(a_2)$. From $i_RRf$, it follows that $i_R
= h \circ f$ for some $h \in F$. Applying $h$ function on both sides
of $f(a_1) = f(a_2)$, we get $h(f(a_1)) = h(f(a_2))$. So, $i_R(a_1) =
i_R(a_2)$. So, $a_1 = a_2$. So, $f$ is one to one.

($\Leftarrow$) Suppose $f$ is one to one. Let $A = Ran(f)$. Let $h =
f^{-1} \cup ((R \setminus A) \times \{0\})$. Proof that $h$ is a
function:

Let $a$ be an arbitrary element in $R$.
Existence proof. Since $f$ is one to one, it follows that $(a,b) \in
f$ for some $b$. So, $(b,a) \in f^{-1}$. So, $(b,a) \in h$.
Uniqueness proof. Let $(b,a_1) \in h$ and $(b,a_2) \in h$. Let us
consider the cases:

a) $b \in A$. Then it follows that $\exists c \in R$ such that $(c,b)
\in f$. So, $(b,c) \in f^{-1}$. Then it is the case that $c = a_1 =
a_2$.
b) $b \notin A$ Then it follows that $(b,0) \in h$. So, $a_1 = a_2 =
0$. So, $h$ is a function.

Proof of $i_R = h \circ f$
\begin{align*}
  h \circ f(a) = h(f(a)) \\
  = h(b) \\
  = a \\
  = i_R(a)  
\end{align*}

\subsection{Solution (e)}
Suppose $g \in F$ and $g$ is a constant function. Let $f$ be an
arbitrary element in $F$. We know that $g = g \circ f$. So, $(g,f) \in
R$. So, $\forall f \in F(gRf)$.

\subsection{Solution (f)}
Suppose $g \in F$ is a constant function. Let $f$ be an arbitrary
element in $F$.
($\Leftarrow$) Suppose $f$ is a constant function. It follows from (e)
that $\forall x \in F(fRx)$. So, $fRg$.

($\Rightarrow$) Suppose $fRg$. Then it follows that $f = h \circ g$
for some $h \in F$. Now $h \circ g$ is a constant function, since $g$
is a constant function. So, $f$ is a constant function.

\subsection{Solution (g)}
Proof: Set of all one to one functions from $R$ to $R$ is the largest
element of $F/S$ in the partial order $T$.

Let $[a]_S$ be the set of all one to one function from $R$ to $R$. We
have to prove $\forall [b] \in F/S ([b]_S T [a]_S)$.

Let $[b]_S$ be an arbitrary element in $F\S$. From (d), it follows
that $i_RRa$. We also know from (c), that $bRi_R$. Since $R$ is
transitive, $bRa$. From $\forall f \in F \forall g \in F([f]_S T [g]_s
\iff fRg)$, it follows that $[b]_S T [a]_S$. Since $[b]_S$ is an
arbitrary element, we can conclude that $[a]_S$ be the largest element
of $F/S$ on partial order $T$.

Proof. Set of all constant functions from $R$ to $R$ is the smallest
element.

We need to prove $\forall [b] \in F/S([a]_S T [b]_S)$ where $[a]_S$ is
the set of all constant functions.

Suppose $[a]_S$ is the set of all constant functions. Let $[b]_S$ be
an arbitrary element in $F/S$. From
$\forall f \in F \forall g \in F([f]_S T [g]_s \iff fRg)$ it follows
that $aRb$. Since $[b]_S$ is arbitrary, we can conclude that $[a]_S$
is the smallest element in the partial order $T$.

\end{document}
